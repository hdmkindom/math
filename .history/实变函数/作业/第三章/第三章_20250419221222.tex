\documentclass[a4paper]{article}
\usepackage[UTF8]{ctex}
\usepackage{amssymb}
\usepackage{amsmath}
\usepackage[margin=2.5cm]{geometry}  % 缩小边距

\title{实变函数课本习题集}
\author{刘泽博}
\date{\today}

\begin{document}
    \maketitle
    \tableofcontents
    \section{第三章练习题}    
    \subsection{第一题}

    f(x),g(x)为E上的可测函数,证明 E(f>g)为可测集?

    prove:
    \[\because f,g\text{为可测函数}\]
    \[\therefore \forall \alpha >0\]
    \[E(f>\alpha)\text{为可测函集,即}E(f>\alpha)=\{x\in E | f(x)>\alpha\}\]
    \[E(g>\alpha)\text{为可测函集,即}E(g>\alpha)=\{x\in E | g(x)>\alpha\}\]
    \[\text{对}E(g>\alpha)\text{来说,}E(g\ge\alpha)=E(g>\alpha)^C\]
    \[\therefore E(g\ge\alpha)\text{为可测集}\]
    \[\forall x \in E(f>g)\]
    \[\exists \alpha\]
    \[s.t. \quad x\in E(f>\alpha) \cap E(g \ge \alpha)\]
    \[\therefore E(f>g)=\bigcup_{r\in \mathbb{Q}}(E(f>\alpha)\cap E(g\ge \alpha))\]
    \[\because E(f>\alpha)\cap E(g\ge\alpha)\text{可测}\]
    \[\therefore \bigcup_{r\in \mathbb{Q}}(E(f>\alpha)\cap E(g\ge \alpha))\text{可测}\]
    \[\therefore E(f>g)\text{可测}\]
    \hfill $\square$

    \subsection{四月18 小习题1}
    
    f 定义在可测集 E 上的广义时函数

    若 $f^2$ 可测,且 $E(f>0)$ 可测

    $Q_1$: 证明 f 可测?

    $Q_2$ : 证明 f 可测 $\Leftrightarrow$ $f^3$ 可测?

    prove;

    \[\text{欲证} E(f \le 0)=\{x|x\in E,f(x) \le 0\}\text{可测}\]
    \[\therefore \forall x \in E\]
    \[f(x)\le 0\]
    \[f^2(x)\ge 0\]
    \[\because f^2 \text{为可测函数}\]
    \[\therefore E(f^2\ge 0)\text{可测}\]
    \[\therefore E(f\le 0)\subset E(f^2\ge 0)\]
    \[\therefore E(f\le 0)\text{可测}\]

    
\end{document}