\documentclass[a4paper]{article}
\usepackage[UTF8]{ctex}
\usepackage{amssymb}
\usepackage{amsmath}
\usepackage[margin=2.5cm]{geometry}  % 缩小边距

\title{实变函数课本习题集}
\author{刘泽博}
\date{\today}

\begin{document}
    \maketitle
    \tableofcontents
    \section{第一章练习题}
    \section{try}
    \subsection{第七题}
    作下列各集的一一对应:

    (2) [a,b]与$(-\infty,\infty)$
    
    尝试作映射使得

    \[[a,b] \sim (-\frac{\pi}{2},\frac{\pi}{2}) \sim (-\infty,\infty)\]
    \[\text{分解映射为}f\circ g\]
    \[f:[a,b] \sim (-\frac{\pi}{2},\frac{\pi}{2})\]
    \[\text{作线性映射f(x),用待定系数法求解,得}\]
    \[f(x)=\frac{\pi}{b-a}x+\frac{(a+b)\pi}{2(a-b)}\]
    \[g:(-\frac{\pi}{2},\frac{\pi}{2}) \sim (-\infty,\infty)\]
    \[g(x)=tanx\]
    \[f\circ g [a,b] =(-\infty,\infty)\]
    \hfill $\square$

    (2) 开区间(0,1)与无理数集

    记无理数集为$\mathbb{R}\backslash\mathbb{Q}$

    尝试作映射使得
    \[(0,1) \sim (0,\infty) \sim \mathbb{R}\backslash\mathbb{Q}\]
    \[\text{分解映射为}f\circ g\]
    \[f:(0,1) \sim (0,\infty)\]
    \[f(x)=\frac{1}{x}-1\]
    \[g:(0,\infty) \sim \mathbb{R}\backslash\mathbb{Q}\]
    \[(0,\infty)\text{中,有理数可列,记为} \{q_1,q_2\ldots q_n \ldots\}\]
    \[
    g(x) =
    \begin{cases}
    x, & x \not\in \mathbb{Q} \\
    q_n+\frac{\pi}{n}, & x = q_n
    \end{cases}
    \]
    \[\text{保证g(x)在无理数点为无理数,有理数点也为无理数}\]
    \[f\circ g (0,1)= \mathbb{R}\backslash\mathbb{Q}\]
    \hfill $\square$

    \subsection{第九题}
    以有理数为端点的区间集能否与自然数集或区间[0,1]构成一一对应?

    能

    证明如下:
    \[\mathbb{Q}\text{为可列集,记为} \{q_1,q_2\ldots q_n \ldots\}\]
    \[\text{区间集A}=\{A_{1,2},A_{2,3}\ldots A_{n,n+1} \ldots\},\text{其中,}A_{n,n+1}\text{为以}q_1,q_2\text{为端点的区间}\]
    \[\text{易知}A可列\]
    \[\therefore A \sim \text{自然数集}\]
    \hfill $\square$

    \subsection{第十题}
    第十题 证明整系数多项式的全体是可列的

    记整系数多项式的全体为
    \[P[x]=\{f(x)=a_0x^n+a_1x^{n-1}+\ldots+a_n|a_i \in \mathbb{Z},n\in \mathbb{N_+}\}\]
    
    记$P[x]$中一真子集为
    \[P'[x]=\{f(x)=a_0x^n+a_1x^{n-1}+\ldots+a_n|a_i \in \mathbb{N_+},n\in \mathbb{N_+}\}\]
    \[\text{现证明}P'[x]\text{可列}\]
    \[P'[x] = \bigcup_{n=1}^{\infty} P'_n[x]\]
    \[\because  P'_n[x]\sim \mathbb{N}\]
    \[\therefore P'_n[x]\text{可列}\]
    \[\therefore P'[x]\text{可列}\]
    \[\therefore P[x]\text{中真子集可列,进而}P[x]\text{可列}\]
    \hfill $\square$

    \subsection{第十六题}
    证明任何点集的内点的全体是开集

    记任意点集为$A$如果一个内点$t_i \in A$
    \[\text{则有}\exists (a_i,b_i),\text{使得}t_i \in (a_i,b_i)\]
    \[(a_i,b_i) \subset A\]
    \[\therefore \forall x \in (a_i,b_i)\text{均有x为内点}\]
    \[\therefore (a_i,b_i)\text{为开集}\]
    \[\text{对于内点}t_j\not\in (a_i,b_i)\]
    \[\text{则有}\exists (a_j,b_j)\text{使得}t_j \in (a_j,b_j) \subset A\]
    \[\text{同理可证}t_j\text{所在的}(a_j,b_j)\text{为开集}\]
    \[\text{由此可知,A内点全体为开集}\]
    \hfill $\square$

    \subsection{第十七题}
    设f(x)是定义在$\mathbb{R}^1$上只取整数值的函数,证明它的连续点集是开集,且不连续点集是闭集

    记$S$为f(x)的连续点集
    
    $T$为f(x)的不连续点集

    1. 证明连续点集是开集
    \[\forall x_0 \in S\]
    \[\exists \delta >0\]
    \[\text{使得}(x_0-\delta,x_0+\delta)\subset S\land f(x)\text{在}x_0\text{处连续}\]
    \[\therefore \lim_{x\to x_0}F(x)=f(x_0)\]
    \[\forall x \in (x_0-\delta,x_0+\delta)\]
    \[\text{有}f(x)=f(x_0)\]
    \[\therefore S\text{为开集}\]

    2. 证明不连续点集是闭集
    \[T=\mathbb{R}^1-S\]

    即证明 如果$\{x_n\}\subset T$
    \[\lim_{n \to \infty}x_n \to x_0\]
    \[\text{有}x_0\in T\]
    \[\text{设}\{x_n\}\subset T\]
    \[\lim_{n \to \infty}x_n \to x_0\]

    反证:
    \[\text{若}x_0\not\in T\]
    \[\text{则}x_0\in S\]
    \[\text{则}\lim_{x\to x_0}f(x)=f(x_0)\]
    \[\exists \delta > 0\]
    \[\forall x \in (x_0-\delta,x_0+\delta)\]
    \[\therefore f(x)=f(x_0)\text{为一定值}\]
    \[\text{然而}\]
    \[\because x_n \in T\]
    \[f(x_n)不连续\]
    \[\therefore \forall \delta >0\text{且充分小}\]
    \[\text{必然}\exists  x \in (x_0-\delta,x_0+\delta)\]
    \[\text{使得}f(x) \ne f(x_0)\]
    \[\text{与} f(x)=f(x_0) \text{矛盾}\]
    \[\therefore x_0 \in T , T \text{为闭集}\]\
    \hfill $\square$

    \subsection{第十八题}
    设点集列$\{E_k\}$是有限区间[a,b]的渐缩列:$E_1 \supset E_2 \supset \ldots,$且每个$E_k$均为非空闭集,证明$\bigcap_{k=1}^{\infty}E_k$非空,试着证明交集$ \bigcap_{k=1}^{\infty} E_k$非空

    \[\text{令}E_1=[a_1,b_1]\subset [a,b]\]
    \[E_2 =[a_2,b_2]\subset E_1\]
    \[E_k = [a_k,b_k]\subset E_{k-1}\]
    \[\text{有}a_1<a_2<\ldots<a_k<\ldots\]
    \[b_1>b_2>\ldots>b_k>\ldots\]
    \[\text{构成区间列}\]
    \[\therefore \exists t_0 \text{为聚点}\]
    \[t_0 \in E_n,n=1,2,3,\ldots\]
    \[\therefore t_0 \in \bigcap_{k=1}^{\infty}E_k\]
    \[\therefore \bigcap_{k=1}^{\infty}E_k\text{非空}\]
    \hfill $\square$

    \section{第二章练习题}
    \subsection{第三题}
    设$G_1,G_2$是开集,且$G_1$是$G_2$的真子集,是否一定有$mG_1<mG_2$?

        \[\because G_1 \subsetneq G_2\]
        \[\therefore \exists x_0 \in G_2 \text{并且}  x_0 \notin G_1\]
        \[\because G_1,G_2 \text{为开集}\]
        \[\therefore \exists \delta > 0,(x_0 - \delta,x_0 + \delta) \subset G_2,(x_0 - \delta,x_0 + \delta)\not\subset G_1\]
        \[\therefore G_1\cup(x_0 - \delta,x_0 + \delta) \subset G_2\]
        \[\therefore mG_1 + m(x_0 - \delta,x_0 + \delta) \leq mG_2\]
        \[\therefore mG_1 \leq mG_2 - 2\delta\]
        \[\therefore mG_1<mG_2 \text{成立}\]
    \hfill $\square$

    \subsection{第四题}
    对任意开集$G$,是否有$mG_1<mG_2$?

        1. 证明: $G \subset G'$
        \[\because G \text{为开集}\]
        \[\therefore \forall x \in G\]
        \[\exists \delta > 0,\text{使得}(x-\delta,x+\delta)\subset G\]
        \[\therefore x \text{为聚点,而非;孤立点}\]
        \[\therefore x \in G'\]
        \[\therefore G \subset G'\]

        2. 证明: $G' \subset G$
        \[\because G'\text{为} G \text{的导集}\]
        \[\therefore \forall x \in G'\]
        \[x \text{必为聚点}\]
        \[\therefore x \in G\]
        \[\therefore G' \subset G\]

        由上可知,如果G为开集,则$G = G'$
        \[\therefore G \text{ 无孤立点,故 } \overline{G} = G \cup G' = G\]

        \[\therefore m\overline{G}=mG\]
    \hfill $\square$

    \subsection{第五题}
    如果把外测度定义改为:"有界集E的外测度定义为包含E的闭集的测度的下确界",是否合理?

        不合理.

        举一反例:定义区间 $[0,1]$ 内的有理数集:
        \[Q_1=\mathbb{Q}\cap[0,1]\]
        \[\text{有}m^{*}Q_1=\underset{Q_1\subset F}{inf}mF=m[0,1]=1\]
        \[mQ=0\]
        \[Q_1\subset Q\]
        \[mQ_1=0\]
        \[m^{*}\ne mQ\]
        \[\therefore \text{该定义不成立}\]
    \hfill $\square$

    \subsection{第六题}
    设$A_1,A_2,\ldots,A_n$是n个互不相交的可测集,且$E_k \subset A_k,k=1,2,\ldots ,n$证明:
    \[m^*(\bigcup_{k=1}^{n}E_k)=\sum_{k=1}^{n}m^*E_k\]

    prove:
    由 外测度的半可加行性可知:
    \[m^*(\bigcup_{k=1}^{n}E_k) \le \sum_{k=1}^{n}m^*E_k\]
    
    现,证明:
    \[m^*(\bigcup_{k=1}^{n}E_k) \ge \sum_{k=1}^{n}m^*E_k\]

    \[\because A_1,A_2,\ldots,A_n \text{是n个互不相交的可测集,且}E_k \subset A_k,k=1,2,\ldots ,n\]
    \[\therefore E_1,E_2,\ldots ,E_n \text{互不相交}\]

    令E=$\bigcup_{k=1}^{n}E_k$,有
    \[m^*(\bigcup_{k=1}^{n}E_k)=inf\{mG|G\text{为开集},E\subset G\}\]
    \[\therefore (G\cap A_k) \supset E_k \supset (G\cap E_k)\]
    \[\therefore mG \ge m\bigcup_{k=1}^{n}(G\cap A_k)=\sum_{k=1}^{n}m(G \cap A_k) = \sum_{k=1}^{n}m^*(G \cap A_k) \ge \sum_{k=1}^{n}m^*E_k\]
    \[\therefore m^*(\bigcup_{k=1}^{n}E_k) \ge \sum_{k=1}^{n}m^*E_k\]
    \hfill $\square$

    \subsection{第七题}
    如果把外测度定义改为"$m^{*}E$为包含E的可测集的测度的下确界",问此定义与原来的外测度定义有何关系?

        两种定义等价

        证明如下:

        记命题 a = 原定义,即$m^{*}E=\underset{G\subset E}{inf}mG$,G为开集

        记命题 b = 新定义,即$m^{*}E=\underset{F\subset E}{inf}mF$,F为可测集

        1. 证明 a -> b

        \[\because G \text{为开集}\]
        \[\therefore G \text{为可测集}\]
        \[\therefore b\text{成立}\]

        2. 证明 b -> a
        \[令\underset{F\subset E}{inf}mF = t\]
        \[\forall \varepsilon >0,\exists \text{可测集}F_1 \supset E=\underset{F\subset E}{inf}mF\]
        \[\text{使得}mF_1<t+\frac{\varepsilon}{2}\]
        \[\because F_1 \text{可测},\exists \text{开集}G_1,\text{使得}G_1\supset F_1\supset E\]
        \[\therefore \text{可以有}mG_1 \le mF_1 + \frac{\varepsilon}{2} \le t + \varepsilon\]
        \[\therefore \text{开集}G_1 \le \underset{F\subset E}{inf}mF + \varepsilon\]
        \[\therefore a\text{成立}\]
    \hfill $\square$

    \subsection{第八题}
    设${E_k}$为$\mathbb{R}$中互不相交的集列$E=\bigcup_{k=1}^{\infty}E_K$,证明:
    \[m_{*}E\geq \sum_{k=1}^{\infty}E_k\]

        由内测度定义可知:   
        \[m_{*}E=\sup_{F\subset E}mF\]
        \[m_{*}E_k=\sup_{F_k\subset E_k}mF_K\]
        \[\therefore \exists F,F_k \text{使得}m_{*}E=mF,m_{*}E_k=mF_k\]
        下面证明$\cup_{k=1}^{\infty}F_k\subset F$
        \[\because F_k \subset E_k\]
        \[E_k=\cup E_k\]
        \[\therefore \cup_{k=1}^{\infty}F_k\subset E\]
        \[\therefore \cup_{k=1}^{\infty}F_k\subset F\]
        \[m\cup_{k=1}^{\infty}F_k = \sum_{k=1}^{\infty}mF_k\]
        \[\sum_{k=1}^{\infty}mF_k \le mF\]
        \[\therefore m_{*}E\geq \sum_{k=1}^{\infty}E_k\]
    \hfill $\square$

    \subsection{第九题}
    设$E_1,E_2$均为有界可测集,证明:
    \[m(E_1\cup E_2)=mE_1+mE_2-m(E_1 \cap E_2)\]

    prove:
    \[\because E_1,E_2 \text{有界且可测}\]
    \[\therefore (E_1 \cup E_2) = E_1 \cap (E_2 - (E_1 \cap E_2))\]
    \begin{align*}
        &\therefore m(E_1 \cup E_2)\\
        & = m(E_1 \cap (E_2 - (E_1 \cap E_2)))\\
        & = m(E_1) + m(E_2 - (E_1 \cap E_2))\\
        & = m(E_1) + m(E_2) - m(E_1 \cap E_2)
    \end{align*}
    \hfill $\square$

    \subsection{第十题}
    设E为$\mathbb{R}$中的可测集,A为任意集,证明
    \[m^{*}(E\cup A)+ m^*(E\cap A)=mE+m^{*}A\]

    当E不可测时,上式如何.

    prove: 
    \[A\cup E \cap E = E\]
    \[A\cup E \cap E^c = A\cap E^c\]
    \[m^*E = mE\]

    由 Caratheodory 条件:A为任意集,E为$\mathbb{R}$中的可测集

    \[m^*A=^*(A\cap E) + m^*(A\cap E^c)\]
    \[m^*(A\cup E)=m^*(A\cup E \cap E) + m^*(A\cup E \cap E^c)\]
    \[m^*(A\cup E)=m^*E+m^*(A\cap E^c)\]
    \[\therefore m^{*}(E\cup A)+ m^*(E\cap A)=m^*E+m^{*}A\]
    \[\therefore m^{*}(E\cup A)+ m^*(E\cap A)=mE+m^{*}A\]
    \hfill $\square$

    \subsection{第十一题}
    设{$E_n$}为[0,1]中的集列,满足:
    \[\sum_{n=1}^{\infty}m^*E_n=\infty\]
    问是否有$m*(\overline{\lim_{n}}E_n)>0$?

    \subsection{第十二题}
    设$E$为可测集,问下列是否成立?

    (i)$m\overline{E}=mE$
    
    (ii)$mE^{\circ}=mE$

    (i)不成立,反例:$E=[0,1]\cap \mathbb{Q}$
    \[\therefore \overline{E}=[0,1]\]
    \[\therefore mE = 0\]
    \[\therefore m\overline{E}=1\]
    \[m\overline{E}\ne mE\]

    (ii)不成立,反例:$E=[0,1] - \mathbb{Q}$
    \[\therefore E^{\circ} = \emptyset\]
    \[\therefore mE^{\circ} = 0\]
    \[\therefore mE = 1\]
    \[mE^{\circ} \ne mE\]

    \subsection{第十四题}
    设$E_1\subset E_2 \subset \cdots E_n \subset \cdots$,试着证明
    \[m^{*}(\bigcup_{n=1}^{\infty}E_n)= \lim_{x\to \infty} m^{*}E_n\]

    prove:

    数学归纳法:得到$m^{*}(\bigcup_{n=1}^{n}E_n)=m^{*}E_n$:

    1. 当n=1时,显然成立.

    2. 当n=2时,有
    \[\because {E_k}\text{递增}\]
    \[\therefore m^{*}(E_1 \cup E_2)=m^*E_2\]

    3. 假设当n=k时成立,即
    \[m^{*}(\bigcup_{n=1}^{k}E_n)=m^{*}E_n\]

    4. 当n=k+1时,有
    \[m^{*}(\bigcup_{n=1}^{k}E_n \cup E_{k+1})=m^{*}(E_n\cup E_{k+1})=m^{*}E_{k+1}\]
    
    均成立

    \[\therefore m^{*}(\bigcup_{n=1}^{n}E_n)=m^{*}E_n\]

    两侧去极限,有
    \[\therefore \lim_{n\to \infty} m^{*}(\bigcup_{n=1}^{n}E_n)=\lim_{n\to \infty} m^{*}E_n\]
    \[m^{*}(\bigcup_{n=1}^{\infty}E_n)= \lim_{x\to \infty} m^{*}E_n\]
    \hfill $\square$


    \subsection{第二十题}
    试作一个闭集 $F\subset[0,1]$,使得$F$中不含任何开区间,而$mF=1/2$

        思路类似构造一个Cantor集,不过取得区间不一样

        首先,已知$\sum_{n=1}^{\infty} \frac{1}{2^{n+1}} = \frac{1}{2}$
        而且,令$G=(0,1)-F$

        $mF=m((0,1)-G)=1-mG$
        
        所以,在尝试构造Cantor集F的时候,扣去的开区间应该符合级数$\sum_{n=1}^{\infty} \frac{1}{2^{n+1}}$

        \[\text{令}F_0=[0,1],\text{记剩余的区间数量和为n}\]
        \[\text{对}F_0\text{四等分,去除中间的}\frac{1}{4}\]
        \[\text{得到}F_1,mG=\frac{1}{4}\]
        \[\text{对}F_1\text{剩余的两个区间,各自去除中间的}(\frac{1}{4})^2\]
        \[\text{得到}F_2,mG=\frac{1}{4} + (\frac{1}{4})^2\times2\]

        \text{以此类推}\ldots

        得到F,并且扣去的G有$mG=\sum_{n=1}^{\infty} \frac{1}{2^{n+1}} = \frac{1}{2}$
        \[\text{因此},mF=\frac{1}{2},\text{且不含任何开区间}\]
    \hfill $\square$
\end{document}